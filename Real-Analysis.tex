% Options for packages loaded elsewhere
\PassOptionsToPackage{unicode}{hyperref}
\PassOptionsToPackage{hyphens}{url}
%
\documentclass[
]{book}
\usepackage{amsmath,amssymb}
\usepackage{iftex}
\ifPDFTeX
  \usepackage[T1]{fontenc}
  \usepackage[utf8]{inputenc}
  \usepackage{textcomp} % provide euro and other symbols
\else % if luatex or xetex
  \usepackage{unicode-math} % this also loads fontspec
  \defaultfontfeatures{Scale=MatchLowercase}
  \defaultfontfeatures[\rmfamily]{Ligatures=TeX,Scale=1}
\fi
\usepackage{lmodern}
\ifPDFTeX\else
  % xetex/luatex font selection
\fi
% Use upquote if available, for straight quotes in verbatim environments
\IfFileExists{upquote.sty}{\usepackage{upquote}}{}
\IfFileExists{microtype.sty}{% use microtype if available
  \usepackage[]{microtype}
  \UseMicrotypeSet[protrusion]{basicmath} % disable protrusion for tt fonts
}{}
\makeatletter
\@ifundefined{KOMAClassName}{% if non-KOMA class
  \IfFileExists{parskip.sty}{%
    \usepackage{parskip}
  }{% else
    \setlength{\parindent}{0pt}
    \setlength{\parskip}{6pt plus 2pt minus 1pt}}
}{% if KOMA class
  \KOMAoptions{parskip=half}}
\makeatother
\usepackage{xcolor}
\usepackage{longtable,booktabs,array}
\usepackage{calc} % for calculating minipage widths
% Correct order of tables after \paragraph or \subparagraph
\usepackage{etoolbox}
\makeatletter
\patchcmd\longtable{\par}{\if@noskipsec\mbox{}\fi\par}{}{}
\makeatother
% Allow footnotes in longtable head/foot
\IfFileExists{footnotehyper.sty}{\usepackage{footnotehyper}}{\usepackage{footnote}}
\makesavenoteenv{longtable}
\usepackage{graphicx}
\makeatletter
\def\maxwidth{\ifdim\Gin@nat@width>\linewidth\linewidth\else\Gin@nat@width\fi}
\def\maxheight{\ifdim\Gin@nat@height>\textheight\textheight\else\Gin@nat@height\fi}
\makeatother
% Scale images if necessary, so that they will not overflow the page
% margins by default, and it is still possible to overwrite the defaults
% using explicit options in \includegraphics[width, height, ...]{}
\setkeys{Gin}{width=\maxwidth,height=\maxheight,keepaspectratio}
% Set default figure placement to htbp
\makeatletter
\def\fps@figure{htbp}
\makeatother
\setlength{\emergencystretch}{3em} % prevent overfull lines
\providecommand{\tightlist}{%
  \setlength{\itemsep}{0pt}\setlength{\parskip}{0pt}}
\setcounter{secnumdepth}{5}
\usepackage{booktabs}
\usepackage{amsthm}
\makeatletter
\def\thm@space@setup{%
  \thm@preskip=8pt plus 2pt minus 4pt
  \thm@postskip=\thm@preskip
}
\makeatother
\ifLuaTeX
  \usepackage{selnolig}  % disable illegal ligatures
\fi
\usepackage[]{natbib}
\bibliographystyle{apalike}
\usepackage{bookmark}
\IfFileExists{xurl.sty}{\usepackage{xurl}}{} % add URL line breaks if available
\urlstyle{same}
\hypersetup{
  pdftitle={Real analysis},
  pdfauthor={Zijie Xia},
  hidelinks,
  pdfcreator={LaTeX via pandoc}}

\title{Real analysis}
\author{Zijie Xia}
\date{2024-08-01}

\usepackage{amsthm}
\newtheorem{theorem}{Theorem}[chapter]
\newtheorem{lemma}{Lemma}[chapter]
\newtheorem{corollary}{Corollary}[chapter]
\newtheorem{proposition}{Proposition}[chapter]
\newtheorem{conjecture}{Conjecture}[chapter]
\theoremstyle{definition}
\newtheorem{definition}{Definition}[chapter]
\theoremstyle{definition}
\newtheorem{example}{Example}[chapter]
\theoremstyle{definition}
\newtheorem{exercise}{Exercise}[chapter]
\theoremstyle{definition}
\newtheorem{hypothesis}{Hypothesis}[chapter]
\theoremstyle{remark}
\newtheorem*{remark}{Remark}
\newtheorem*{solution}{Solution}
\begin{document}
\maketitle

{
\setcounter{tocdepth}{1}
\tableofcontents
}
\chapter*{Preface}\label{preface}
\addcontentsline{toc}{chapter}{Preface}

This is a brief note of \emph{Real Analysis} by Elias M. Stein \& Rami Shakarchi.

\chapter{Measure Theory}\label{ch1}

\section{Preliminaries}\label{preliminaries}

The \textbf{open ball} in \(\mathbb{R}^d\) centered at \(x\) and of radius \(r\) is defined by
\[
B_r(x)=\{y\in \mathbb{R}^d :|y−x|<r\}.
\]
A subset \(E\subset\mathbb{R}^d\) is \textbf{open} if for every \(x\in E\) there exists \(r>0\) with \(B_r(x)\subset E\). By definition, a set is \textbf{closed} if its complement is open.

We note that any (not necessarily countable) union of open sets is open, while in general the intersection of only finitely many open sets is open. A similar statement holds for the class of closed sets, if one interchanges the roles of unions and intersections.

A set \(E\) is \textbf{bounded} if it is contained in some ball of finite radius. A bounded set is \textbf{compact} if it is also closed. Compact sets enjoy the \textbf{Heine-Borel covering property}:

\begin{itemize}
\tightlist
\item
  Assume \(E\) is compact, \(E\subset \bigcup_{\alpha}\cal{O}_{\alpha}\), and each \(\cal{O}_{\alpha}\) is open. Then there are finitely many of the open sets, \(\cal{O}_{\alpha_1},{O}_{\alpha_2},\dots,{O}_{\alpha_N}\), such that \(E\subset \bigcup_{j=1}^n\cal{O}_{\alpha_j}\).
\end{itemize}

In words, any covering of a compact set by a collection of open sets contains a finite subcovering.

A point \(x\in \mathbb{R}^d\) is a \textbf{limit point} of the set \(E\) if for every \(r>0\),the ball \(B_r(x)\) contains points of \(E\). This means that there are points in \(E\) which are arbitrarily close to \(x\). An \textbf{isolated point} of \(E\) is a point \(x\in E\) such that there exists an \(r > 0\) where \(B_r(x)\cap E\) is equal to \(\{x\}\).

A point \(x\in E\) is an \textbf{interior point} of \(E\) if there exists \(r>0\) such that \(B_r(x)\subset E\). The set of all interior points of E is called the \textbf{interior} of \(E\). Also, the \textbf{closure} \(\overline{E}\) of the \(E\) consists of the union of \(E\) and all its \emph{limit points}. The \textbf{boundary} of a set \(E\), denoted by \(\partial E\), is the set of points which are in the closure of \(E\) but not in the interior of \(E\).

Note that the closure of a set is a closed set; every point in \(E\) is a limit point of \(E\); and a set is closed if and only if it contains all its limit points. Finally, a closed set \(E\) is \textbf{perfect} if \(E\) does not have any isolated points.

\begin{lemma}
If a rectangle is the almost disjoint union of finitely many other rectangles, say \(R=\bigcup_{k=1}^NR_k\), then
\[
|R|=\sum_{k=1}^N |R_k|
\]
\end{lemma}

\begin{lemma}
If \(R,R_1,\dots,R_N\) are rectangles, and \(R\subset\bigcup_{k=1}^NR_k\), then
\[
|R|\leq\sum_{k=1}^N |R_k|
\]
\end{lemma}

\emph{Proof.}

Every open subset \(\cal{O}\) of \(\mathbb{R}\) can be written uniquely as a countable union of disjoint open intervals.

Every open subset \(\cal{O}\) of \(\mathbb{R}^d\), \(d\geq 1\), can be written uniquely as a countable union of almost disjoint closed cubes.

The \textbf{Cantor set} \(\cal{C}\) is by definition the intersection of all \(C_k\)'s:
\[
\mathcal{C}=\bigcap_{k=0}^{\infty}C_k.
\]

The set \(\cal{C}\) is not empty, since all end-points of the intervals in \(C_k\) (all \(k\)) belong to \(\cal{C}\).
Despite its simple construction, the Cantor set enjoys many interesting topological and analytical properties. For instance, \(\cal{C}\) is closed and bounded, hence compact. Also, \(\cal{C}\) is totally disconnected: given any \(x,y\in \cal{C}\) there exists \(z\not\in \cal{C}\) that lies between \(x\) and \(y\). Finally, \(\cal{C}\) is perfect: it has no isolated points.

In terms of cardinality the Cantor set is rather large: it is not countable. Since it can be mapped to the interval \([0,1]\), the Cantor set has the cardinality of the continuum. However, from the point view of ``length'' the size f \(\cal C\) is small. Roughly speaking, it has length zero.

\section{The exterior measure}\label{the-exterior-measure}

\begin{definition}
If \(E\) is any subset of \(\mathbb{R}^d\), the \textbf{exterior measure} of \(E\) is
\[
m_{\ast}(E) = \inf\sum_{j=1}^{\infty}|Q_j|,
\]
where the infimum is taken over all countable coverings \(E\subset \bigcup_{j=1}^{\infty}Q_j\) by closed cubes. In general we have \(0\leq m_{\ast}(E)\leq \infty\).
\end{definition}

\begin{itemize}
\tightlist
\item
  It would not suffice to allow finite sums in the definition of \(m_{\ast}(E)\). The quantity that would be obtained if one considered only coverings of \(E\) by finite unions of cubes is in general larger than \(m_{\ast}(E)\).
\item
  One can, however, replace the coverings by cubes, with coverings by rectangles; or with coverings by balls.
\end{itemize}

\begin{example}
The exterior measure of a closed cube is equal to its volume.
\end{example}

\emph{Proof.}

We consider an arbitrary covering \(Q\subset \bigcup_{j=1}^{\infty}Q_j\) by cubes, where \(Q_j\) is closed for \(j=1,2,\dots\). Then it suffices to prove that
\[
  |Q|\leq \sum_{j=1}^{\infty}|Q_j|.
  \]

\begin{example}
The exterior measure of a open cube is equal to its volume.
\end{example}

\begin{example}
The exterior measure of a rectangle is equal to its volume.
\end{example}

\emph{Proof.}

Arguing as in the first example, we see that \(|R|\leq m_{\ast}(R)\). To obtain the reverse inequality, consider a grid in \(\mathbb{R}^d\) formed by cubes of side length \(1/k\), then let \(k\) tend to infinity yields \(m_{\ast}(R)\leq |R|\).

\begin{example}
The Cantor set \(\cal C\) has exterior measure 0.
\end{example}

\subsection{Properties of the exterior measure}\label{properties-of-the-exterior-measure}

From the definition we know that:
- For every \(\epsilon>0\), there exists a covering \(E\subset \bigcup_{j=1}^{\infty}Q_j\) with
\[
\sum_{j=1}^{\infty}m_{\ast}(Q_j)\leq m_{\ast}(E)+\epsilon.
\]

\begin{proposition}
\protect\hypertarget{prp:pem}{}\label{prp:pem}\leavevmode

\begin{enumerate}
\def\labelenumi{\arabic{enumi}.}
\tightlist
\item
  (Monotonicity) If \(E_1\subset E_2\), then \(m_{\ast}(E_1)\leq m_{\ast}(E_2)\).
\item
  (Countatble sub-additivity) If \(E= \bigcup_{j=1}^{\infty}E_j\), then \(m_{\ast}(E)\leq\sum_{j=1}^{\infty}m_{\ast}(E_j)\).
\item
  If \(E\subset\mathbb{R}^d\), then \(m_{\ast}(E)=\inf m_{\ast}(\cal O)\), where the infimum is taken over all open sets \(\cal O\) containing \(E\).

  \emph{Proof.}

  It suffices to prove that \(m_{\ast}(E)\geq \inf m_{\ast}(\cal O)\). Notice that for any closed cube \(Q_j\), there exists an open set \(S_j\) which contains \(Q_j\) and such that \(|S_j|\leq (1+\epsilon)|Q_j|\) for a fixed \(\epsilon>0\).
\item
  If \(E=E_1\cup E_2\),and \(d(E_1,E_2)>0\), then
  \[
   m_{\ast}(E)=  m_{\ast}(E_1)+  m_{\ast}(E_2).
  \]
\item
  If a set \(E\) is the countable union of almost disjoint cubes \(E=\bigcup_{j=1}^{\infty}Q_j\), then
  \[
  m_{\ast}(E) = \sum_{j=1}^{\infty}|Q_j|.
  \]

  \emph{Proof.}

  Consider constructing a family of cubes that are at a finite distance from one another, so we can use proposition 4.
\end{enumerate}

\end{proposition}

\section{Measurable sets and the Lebesgue measure}\label{measurable-sets-and-the-lebesgue-measure}

The notion of measurability isolates a collection of subsets in \(\mathbb{R}^d\) for which the exterior measure satisfies all our desired properties, including additivity (and in fact countable additivity) for disjoint unions of sets.

There are a number of different ways of defining measurability, but these all turn out to be equivalent. Probably the simplest and most intuitive is the following: A subset \(E\) of \(\mathbb{R}^d\) is \textbf{Lebesgue measurable}, or simply \textbf{measurable}, if for any \(\epsilon > 0\) there exists an open set \(\cal O\) with \(E\subset\cal O\) and
\[
m_{\ast}(\mathcal{O}-E) ≤ \epsilon.
\]
This should be compared to Proposition \ref{prp:pem} 3, which holds for all sets \(E\). If E is measurable, we define its \textbf{Lebesgue measure} (or \textbf{measure}) \(m(E)\) by
\[
m(E)=m_{\ast}(E).
\]

Immediately from the definition, we find:

\begin{proposition}
\leavevmode

\begin{enumerate}
\def\labelenumi{\arabic{enumi}.}
\tightlist
\item
  Every open set in \(\mathbb{R}^d\) is measurable.
\item
  If \(m_{\ast}(E)=0\), then \(E\) is measurable. In particular, if \(F\) is a subset of a set of exterior measure \(0\), then \(F\) is measurable.
\item
  A countable union of measurable sets is measurable.
\item
  Closed sets are measurable.

  \emph{Proof.}

  Since any closed set \(F\) can be written as the union of compact sets, say \(F=\bigcup_{k=1}^{\infty}F\cap B_k\), where \(B_k\) denotes the closed ball of radius \(k\) centered at the origin, it suffices to prove that compact sets are measurable.
\item
  The complement of a measurable set is measurable.

  \emph{Proof.}

  For every positive integer \(n\) we choose an open set \(\cal O_n\) with \(E\subset \cal O_n\) and \(m_{\ast}(\mathcal{O_n}-E)\leq 1/n\). Notice that
  \[
    (E^c-\bigcup_{n=1}^{\infty}\mathcal{O_n^c}) \subset(\mathcal{O_n}-E)
    \]
  then \(E^c\) is measurable since \(E^c=(E^c-\bigcup_{n=1}^{\infty}\mathcal{O_n^c})\cup \bigcup_{n=1}^{\infty}\mathcal{O_n^c}\).
\item
  A countable intersection of measurable sets is measurable.
\end{enumerate}

\end{proposition}

To prove the fourth proposition, we need the following lemma.

\begin{lemma}
If \(F\) is closed, \(K\) is compact, and these sets are disjoint, then \(d(F,K)>0\).
\end{lemma}

\begin{theorem}
If \(E_1,E_2,\dots\), are disjoint measurable sets, and \(E=\bigcup_{j=1}^{\infty}E_j\), then
\[
m(E)=\sum_{j=1}^{\infty}m(E_j).
\]
\end{theorem}

\emph{Proof.}

If \(F_1,F_2,\dots, F_N\), are compact and disjoint, then obviously for any \(j,k,j\ne k\), \(d(F_j,F_k)>0\), so \(m\left(\bigcup_{j=1}^NF_j\right)=\sum_{j=1}^Nm(F_j)\). If each \(E_j\) is bounded, we can choose a closed subset \(F_j\) for \(E_j\) with \(m_{\ast}(E_j-F_j)\leq \epsilon/2^j\) for each \(j\). Then
\[
  m(E) \geq \sum_{j=1}^Nm(F_j) \ge \sum_{j=1}^Nm(E_j)-\epsilon
  \]
Letting \(N\) tend to infinity, since \(\epsilon\) is arbitrary we find that
\[
  m(E)\ge\sum_{j=1}^{\infty}m(E_j).
  \]
In the general case, we select any sequence of cubes \(\{Q_k\}_{k=1}^{\infty}\) that increases to \(\mathbb{R}^d\), in the sense that \(Q_k\subset Q_{k+1}\) for all \(k\ge 1\) and \(\bigcup_{k=1}^{\infty}Q_k=\mathbb{R}^d\). We then let \(S_1=Q_1\) and \(S_k=Q_k-Q_{k-1}\) for \(k\ge 2\). If we define measurable sets by \(E_{j,k}=E_j\cap S_k\), then
\[
  m(E) =\sum_{j,k}m(E_{j,k})=\sum_j\sum_km(E_{j,k})=\sum_jm(E_j).
  \]

\begin{corollary}

Suppose \(E_1,E_2,...\) are measurable subsets of \(\mathbb{R}^d\).

\begin{itemize}
\tightlist
\item
  If \(E_k \nearrow E\), then \(m(E) =\lim_{N\to \infty}m(E_N)\).
\item
  If \(E_k \searrow E\) and \(m(E_k)<\infty\) for some \(k\), then \(m(E) =\lim_{N\to \infty}m(E_N)\).
\end{itemize}

\end{corollary}

\begin{theorem}
\protect\hypertarget{thm:m}{}\label{thm:m}

Suppose \(E\) is measurable subset of \(\mathbb{R}^d\). Then for every \(\epsilon>0\):

\begin{enumerate}
\def\labelenumi{\arabic{enumi}.}
\tightlist
\item
  There exists an open set \(\cal O\) with \(E\subset \cal O\) and \(m(\mathcal{O}-E)\leq\epsilon\).
\item
  There exists a closed set \(F\) with \(F\subset E\) and \(m(E-F)\leq\epsilon\).
\item
  If \(m(E)\) is finite, there exists a compact set \(K\) with \(K\subset E\) and \(m(E-K)\leq \epsilon\).
\item
  If \(m(E)\) is finite, there exists a finite union \(F=\bigcup_{j=1}^NQ_j\) of closed cubes such that
  \[
  m(E\triangle F)\leq \epsilon.
  \]

  \emph{Proof.}

  Choose a family of closed cubes \(\{Q_j\}_{j=1}^{\infty}\) so that
  \[
    E\subset \bigcup_{j=1}^{\infty}Q_j \text{ and } \sum_{j=1}^{\infty}|Q_j|\le m(E) +\epsilon/2.
    \]
  Since \(m(E) <\infty\), then series converges and there exists \(N>0\) such that \(\sum_{j=n+1}^{\infty}|Q_j|\le \epsilon/2\). If \(F=\bigcup_{j=1}^NQ_j\), then
  \[
   \begin{aligned}
   m(E\triangle F) &= m(E-F)+m(F-E)\\
   &\le m\left( \bigcup_{j=1}^{\infty}Q_j - F\right)+m\left( \bigcup_{j=1}^{\infty}Q_j-E\right)\\
   &\le m\left( \bigcup_{j=n+1}^{\infty}Q_j\right)+ m\left( \bigcup_{j=n+1}^{\infty}Q_j\right)-m(E)\\
   &\le \sum_{j=n+1}^{\infty}|Q_j|+\sum_{j=1}^{\infty}|Q_j|-m(E)\\
   &\le \epsilon.
   \end{aligned}
   \]
\end{enumerate}

\end{theorem}

\subsection{Invariance properties of Lebesgue measure}\label{invariance-properties-of-lebesgue-measure}

\subsection{\texorpdfstring{\(\sigma\)-algebra and Borel sets}{\textbackslash sigma-algebra and Borel sets}}\label{sigma-algebra-and-borel-sets}

\textbf{Borel \(\sigma\)-algebra} in \(\mathbb{R}^d\), denoted by \(\cal B_{\mathbb{R}^d}\), is the smallest \(\sigma\)-algebra in \(\mathbb{R}^d\) that contains all open sets. Elements of this \(\sigma\)-algebra are called \textbf{Borel sets}. Since we observe that any intersection (not necessarily countable) of \(\sigma\)-algebra is again a \(\sigma\)-algebra, we may define \(\cal B_{\mathbb{R}^d}\) as the intersection of all \(\sigma\)-algebras that contain the open sets. This shows the existence and uniqueness of the Borel \(\sigma\)-algebra.

\begin{remark}
There exists Lebesgue measurable sets that are not Borel sets. (See Exercise @ref(exr: \#35)
\end{remark}

\subsection{Construction of a non-measurable set}\label{construction-of-a-non-measurable-set}

\subsection{Axiom of choice}\label{axiom-of-choice}

\section{Measurable functions}\label{measurable-functions}

\subsection{Definition and basic properties}\label{definition-and-basic-properties}

The starting point is the notion of a \textbf{characteristic function} of a set E, which is defined by
\[
\chi_E(x)=
\begin{cases}
1& \text{if }x\in E,\\
0& \text{if }x\not\in E.
\end{cases}
\]
For the Riemann integral it is in effect theclass of \textbf{step functions} that build the blocks of integration theory , with each give as a finite sum

\[
f = \sum_{k=1}^Na_k\chi_{R_k}
\]
where each \(R_k\) is a rectangle, and the \(a_k\) are constants.

For the lebesgue integral we need a more general notion. A \textbf{simple function} is a finite sum
\[
f = \sum_{k=1}^Na_k\chi_{E_k}
\]
where each \(E_k\) is a measurable set of finite measure, and the \(a_k\) are constants.

A function \(f\) defined on a measurable subset \(E\) of \(\mathbb{R}^d\) is \textbf{measurable}, if for all \(a\in\mathbb{R}\), the set
\[
f^{-1}([-\infty,a))=\{x\in E: f(x)<a\}
\]
is measurable. Note that this definition applies to extended-valued functions, so we use \(f^{-1}([-\infty,a))\) instead of \(f^{-1}((-\infty,a))\).

\begin{proposition}
\leavevmode

\begin{enumerate}
\def\labelenumi{\arabic{enumi}.}
\tightlist
\item
  The finite-valued function \(f\) is measurable iff \(f^{-1}(\cal O)\) is measurable for every open set \(\cal O\), and iff \(f^{-1}(F)\) is measurable for every closed set \(F\). (Note that this property also applies to extended-valued functions, if we make the additional hypothesis that both \(f^{-1}(\infty)\) and \(f^{-1}(-\infty)\) are measurable sets since \([-\infty,a)=\{-\infty\}\cup\left(\bigcup_{n=1}^{\infty}(-n,a)\right)\).)
\item
  If \(f\) is continuous on \(\mathbb{R}^d\), then \(f\) is measurable. If \(f\) is measurable and finite-valued, and \(\Phi\) is continuous, then \(\Phi\circ f\) is measurable. (Note that it is not true that \(f\circ\Phi\) is measurable. See exercise 35.)

  \emph{Proof.}

  \(\Phi\) is continuous, so \(\Phi^{-1}((-\infty,a))\) is open set \(\cal O\).
\item
  Suppose \(\{f_n\}_{n=1}^{\infty}\) is a sequence of measurable functions. Then
  \[
  \sup_nf_n(x),\quad \inf_nf_n(x),\quad\limsup_{n\to\infty}f_n(x)\quad \text{and} \quad\liminf_{n\to\infty}f_n(x)
  \]
  are measurable.
\item
  If \(\{f_n\}_{n=1}^{\infty}\) is a collection of measurable functions, and
  \[
  \lim_{n\to\infty}f_n(x)=f(x),
  \]
  then \(f\) is measurable.
\item
  If \(f\) and \(g\) are measurable, then
\end{enumerate}

\begin{enumerate}
\def\labelenumi{(\roman{enumi})}
\tightlist
\item
  The integer powers \(f^k\), \(k\ge 1\) are measurable.
\item
  \(f+g\) and \(fg\) are measurable if both \(f\) and \(g\) are finite-valued.
\end{enumerate}

\begin{enumerate}
\def\labelenumi{\arabic{enumi}.}
\setcounter{enumi}{5}
\tightlist
\item
  Suppose \(f\) is measurable, and \(f(x)=g(x)\) for a.e. \(x\). Then \(g\) is measurable.
\end{enumerate}

\end{proposition}

\subsection{Approximation by simple functions or step functions}\label{approximation-by-simple-functions-or-step-functions}

\begin{theorem}
\leavevmode

Suppose \(f\) is a non-negative measurable function on \(\mathbb{R}^d\). Then there exists an increasing sequence of non-negative simple functions \(\{\varphi_n\}_{n=1}^{\infty}\) that converges pointwise to \(f\), namely,
\[
\varphi_k(x)\le\varphi_{k+1}(x) \text{ and } \lim_{k\to \infty}\varphi_k(x)=f(x),\ \text{for all }x.
\]

\emph{Proof.}

\(\Phi\) is continuous, so \(\Phi^{-1}((-\infty,a))\) is open set \(\cal O\).

\end{theorem}

\begin{theorem}
\protect\hypertarget{thm:s}{}\label{thm:s}\leavevmode

Suppose \(f\) is a measurable function on \(\mathbb{R}^d\). Then there exists a sequence of simple functions \(\{\varphi_n\}_{n=1}^{\infty}\) that satisfies
\[
|\varphi_k(x)|\le|\varphi_{k+1}(x)| \text{ and } \lim_{k\to \infty}\varphi_k(x)=f(x),\ \text{for all }x.
\]

\emph{Proof.}

We use the decomposition of \(f\): \(f(x)=f^{+}(x)-f^{-}(x)\).

\end{theorem}

\begin{theorem}
\leavevmode

Suppose \(f\) is measurable on \(\mathbb{R}^d\). Then there exists an sequence of step functions \(\{\psi_n\}_{n=1}^{\infty}\) that converges pointwise to \(f(x)\) for almost every \(x\),
\[
\varphi_k(x)\le\varphi_{k+1}(x) \text{ and } \lim_{k\to \infty}\varphi_k(x)=f(x),\ \text{for all }x.
\]

\emph{Proof.}

By Theorem \ref{thm:s}, it suffices to show that if \(E\) is a measurable set with finite measure, then \(f=\chi_E\) can be approximated by step functions. This can be proven by split \(E\) into cubes and then rectangles with Theorem \ref{thm:m}.

\end{theorem}

\subsection{Littlewood's three principles}\label{littlewoods-three-principles}

\begin{theorem}[Egorov]
\protect\hypertarget{thm:egorov}{}\label{thm:egorov}Suppose \(\{f_k\}_{k=1}^{\infty}\) is a sequence of measurable functions defined on a measurable set \(E\) with \(m(E)<\infty\), and assume that \(f_k\to f\) a.e. on \(E\) and \(f_1,f_2,\dots,f_k,f\) are finite valued a.e. on \(E\). Given \(\epsilon>0\), we can find a closed set \(A_{\epsilon}\subset E\) such that \(m(E-A_{\epsilon})\le \epsilon\) and \(f_k\to f\) uniformly on \(A_{\epsilon}\).
\end{theorem}

\begin{remark}
Note that \(f_1,f_2,\dots,f_k,f\) are finite valued a.e. on \(E\). Indeed, if it is not satisfied, then we cannot construct \(\{E_k^n\}_{k=1}^\infty\) such that \(E_k^n\nearrow E\). A counterexample is that
\[
f_k(x)=\begin{cases}
k&|x|\le k,\\
\infty &|x|>k
\end{cases}
\]
and \(f(x)=\infty\) on \(\mathbb{R}\).
\end{remark}

\begin{remark}
Note that \(m(E)<\infty\) and it is easy to construct counterexamples when \(m(E)=\infty\). Indeed, if \(m(E)=\infty\), then we cannot find \(k_n\) such that \(m(E-E_{k_n}^n)<1/2^n\) since \(m(E-E_{k_n}^n)=m(E)-m(E_{k_n}^n)\).
\end{remark}

\begin{theorem}[Lusin]
\protect\hypertarget{thm:lusin}{}\label{thm:lusin}Suppose \(f\) is measurable and finite valued a.e. on \(E\) with \(E\) of finite measure. Then for every \(\epsilon>0\) there exists a closed set \(F_{\epsilon}\), with
\[
F_{\epsilon}\subset E, \text{ and }m(E-F_{\epsilon})\le \epsilon
\]
and such that \(f|_{F_{\epsilon}}\) is continuous.
\end{theorem}

\section{The Brunn-Minkowski inequality}\label{the-brunn-minkowski-inequality}

\begin{remark}
\leavevmode

Let \(a,b\ge0\), then
\[
\begin{aligned}
(a+b)^\gamma&\ge a^\gamma+b^\gamma\text{ if }\gamma\ge 1,\\
(a+b)^\gamma&\le a^\gamma+b^\gamma\text{ if }0<\gamma< 1
\end{aligned}
\]

\emph{Proof.}

Let \(f(\gamma)=(1+x)^\gamma-(1+x^\gamma)\), where \(x> 0\). Then
\[
\begin{aligned}
f^\prime (\gamma)&=(1+x)^\gamma\ln(1+x)-x^\gamma\ln x\\
&=[(1+x)^\gamma-x^\gamma]\ln(1+x)+x^\gamma\ln(1+1/x)>0,
\end{aligned}
\]
notice that \(f(1)=0\), so when \(\gamma\ge 1\), \((1+x)^\gamma\ge(1+x^\gamma)\) and when \(0<\gamma< 1\), \((1+x)^\gamma<(1+x^\gamma)\). With this result, the original inequality is obvious.

\end{remark}

\ref{exr:19}

\begin{theorem}
Suppose \(A\) and \(B\) are measurable sets in \(\mathbb{R}^d\) and their sum \(A+B\) is also measurable. Then
\[
m(A+B)^{1/d}\ge m(A)^{1/d}+m(B)^{1/d}.
\]
\end{theorem}

\section{Exercise}\label{exercise}

\section{Problem}\label{problem}

\chapter{Integration Theory}\label{ch2}

\section{The lebesgue integral: basic properties and convergence theorems}\label{the-lebesgue-integral-basic-properties-and-convergence-theorems}

\subsection{Stage one: simple functions}\label{stage-one-simple-functions}

\begin{proposition}

The integral of simpe functions defined bve satifies the following properties:

\begin{enumerate}
\def\labelenumi{\arabic{enumi}.}
\tightlist
\item
  Independence of the representation. If \(\varphi=\sum_{k=1}^Na_k\chi_{E_k}\) is any representation of \(\varphi\), then
  \[
  \int \varphi=\sum_{k=1}^Na_km(E_k).
  \]
\item
  Linearity.
\item
  Additivity.
\item
  Monotonicity.
\item
  Triangle inequality. If \(\varphi\) is a simple function, then so is \(|\varphi|\), and
  \[
  \left|\int\varphi\right|\le \int|\varphi|.
  \]
\end{enumerate}

\end{proposition}

\subsection{Stage two: bounded functions supported on a set of finite measure}\label{stage-two-bounded-functions-supported-on-a-set-of-finite-measure}

\begin{lemma}
\protect\hypertarget{lem:l}{}\label{lem:l}

Let \(f\) be a bounded function supported on a set \(E\) of finite measure. If \(\{\varphi_n\}_{n=1}^\infty\) is any sequence of simple functions bounded by \(M\), supported on \(E\), and with \(\varphi_n(x)\to f(x)\) for a.e. \(x\), then:

\begin{enumerate}
\def\labelenumi{\arabic{enumi}.}
\tightlist
\item
  The limit \(\lim_{n\to\infty}\int\varphi_n\) exists.
\item
  If \(f=0\) a.e., then the limit \(\lim_{n\to\infty}\int\varphi_n\) equals 0.
\end{enumerate}

\end{lemma}

\emph{Proof.}

Setting \(I_n=\int\varphi_n\) and applying Egorov's theorem which is proven in Chapter \ref{ch1} we have that for and large \(n\) and \(m\)
\[
\begin{aligned}
|I_n-I_m| &\le \int_E|\varphi_n-\varphi_m|\\
&=\int_{A_\epsilon}|\varphi_n-\varphi_m|+\int_{E-A_\epsilon}|\varphi_n-\varphi_m|\\
&\le \int_{A_\epsilon}\epsilon\ dx+\int_{E-A_\epsilon}2M\ dx\\
&\le m(E)\epsilon+2M\epsilon.
\end{aligned}
\]
given any \(\epsilon>0\).
This proves that \(\{I_n\}\) os a Cauchy sequence nd hence converges. If \(f=0\), letting \(m\) tend to infinity we have \(|I_n-f|=|I_n|\le m(E)\epsilon+2M\epsilon\), which yields \(\lim_{n\to\infty}I_n=0\).

For a bounded function \(f\) that is supported on sets of finite measure, we define its \textbf{Lebesgue integral} by
\[
\int f=\lim_{n\to\infty}\int\varphi_n.
\]
where \(\{\varphi_n\}\) is any sequence of simple functions satisfying: \(|\varphi_n|\le M\), each \(\varphi_n\) is supported on the support of \(f\), and \(\varphi_n(x)\to f(x)\) for a.e. \(x\) as \(n\) tends to infinity.

Next, we must show that \(\int f\) is independent of the limiting sequence \(\{\varphi_n\}\) used, in order for the integral to be well-defined. Suppose that \(\{\psi_n\}\) is another sequence of simple functions that satisfies the properties above. Then, if \(\eta_n = \varphi_n- \psi_n\), the sequence \(\{\eta_n\}\) consists of simple functions bounded by \(2M\), supported on a set of finite measure, and such that \(\eta_n\to 0\) a.e. as \(n\) tends to infinity. Applying the lemma we find
\[
\lim_{n\to\infty}\int \varphi_n =\lim_{n\to\infty}\int \psi_n+\lim_{n\to\infty}\int \eta_n=\lim_{n\to\infty}\int \psi_n
\]
as desired.

\begin{proposition}

Suppose \(f\) and \(g\) are bounded functions supported on sets of finite measure. Then the following properties hold.

\begin{enumerate}
\def\labelenumi{\arabic{enumi}.}
\tightlist
\item
  Linearity.
\item
  Additivity.
\item
  Monotonicity.
\item
  Triangle inequality.
\end{enumerate}

\end{proposition}

\begin{theorem}[Bounded convergence theorem]
\protect\hypertarget{thm:b}{}\label{thm:b}\leavevmode

Suppose that \(\{f_n\}\) is a sequence of measurable functions that are all bounded by \(M\), are supported on a set \(E\) of finite measure, and \(f_n(x)\to f(x)\) a.e. \(x\) as \(n\to\infty\). Then \(f\) is measurable, bounded ,supported on \(E\) for a.e. \(x\), and
\[
\int|f_n-f|\to 0 \text{ as } n\to\infty.
\]
Consequently,
\[
\int f_n\to \int f  \text{ as } n\to\infty.
\]

\emph{Proof.}

The proof is a reprise of the argument in Lemma \ref{lem:l}. Given \(\epsilon>0\), we may find, by Egorov' theorem,
\[
\begin{aligned}
\int|f_n-f| &\le \int_{A_\epsilon}|f_n-f|+\int_{E-A_\epsilon}|f_n-f|\\
&\le m(E)\epsilon+2M\epsilon.
\end{aligned}
\]
for all large \(n\).

\end{theorem}

\subsection{Return to Riemann integrable functions}\label{return-to-riemann-integrable-functions}

\begin{theorem}
\leavevmode

Suppose \(f\) os Riemann integrable on the closed interval \([a,b]\). Then \(f\) is measurable, and
\[
\cal \int_{[a,b]}^Rf(x)\ dx=\int_{[a,b]}^Lf(x)\ dx.
\]

\emph{Proof.}

By definition of Riemann integrability, \(f\) is bounded, say \(|f(x)|\le M\), and we may construct two consequences of step functions \(\{\varphi_k\}\) and \(\{\psi_k\}\) that satisfy the following properties: \(|\varphi_k(x)\le M|\) and \(|\psi_k(x)\le M|\) for all \(x\in[a,b]\) and \(k\ge 1\),
\[
\varphi_1(x)\le \varphi_2(x) \le \cdots\le f\le\cdots\le \psi_2(x)\le \psi_1,
\]
and
\[
\lim_{k\to\infty}\cal \int_{[a,b]}^R\varphi_k(x)\ dx=\lim_{k\to\infty}\cal \int_{[a,b]}^R\psi_k(x)\ dx=\cal \int_{[a,b]}^Rf(x)\ dx.
\]
Notice that
\[
\int_{[a,b]}^\mathcal{R}\varphi_k(x)\ dx=\int_{[a,b]}^\mathcal{L}\varphi_k(x)\ dx \text{ and }\cal \int_{[a,b]}^R\psi_k(x)\ dx=\cal \int_{[a,b]}^L\psi_k(x)\ dx.
\]
Consider Theorem \ref{thm:b} (Bounded convergence theorem) and you will complete the proof.

\end{theorem}

\subsection{Stage three: non-negative functions}\label{stage-three-non-negative-functions}

In the case of such a function \(f\) we define its \textbf{Lebesgue integral} by
\[
\int f =\sup_g\int g.
\]

\begin{proposition}

The integral of non-negative measurable functions enjoys the following properties:

\begin{enumerate}
\def\labelenumi{\arabic{enumi}.}
\tightlist
\item
  Linearity.
\item
  Additivity.
\item
  Monotonicity.
\item
  If \(g\) is integrable and \(0\le f\le g\), then \(f\) is integrable.
\item
  If \(f\) is integrable, then \(f(x)<\infty\) for a.e. \(x\).
\item
  If \(\int f=0\), then \(f(x)=0\) for a.e. \(x\).

  \emph{Proof.}

  We just prove the first assertion. Let \(\varphi\), \(\psi\) and \(\eta\) be non-negative functions bounded and supported on sets of finite measure, where \(\varphi\le f\), \(\psi\le g\) and \(\eta\le f+g\), then \(\varphi+\psi\le f+g\). Consequently,
  \[
  \int f+ \int g\le \int (f+g).
  \]
  On the other hand, if we define \(\eta_1=\min(f(x),\eta(x))\) and \(\eta_2=\eta-\eta_1\), then
  \[
  \int \eta=\int(\eta_1+\eta_2)=\int \eta_1+\int \eta_2\le\int f+ \int g,
  \]
  which means that
  \[
  \int (f+g)\le \int f+ \int g.
  \]
\end{enumerate}

\end{proposition}

\begin{lemma}[Fatou]
\protect\hypertarget{lem:fatou}{}\label{lem:fatou}\leavevmode

Suppose \(\{f_n\}\) is a sequence of measurable functions with \(f_n\ge 0\). If \(\lim_{n\to\infty}f_n(x)=f(x)\) for a.e. \(x\), then
\[
\int f\leq \liminf_{n\to\infty}\int f_n.
\]

\emph{Proof.}

Suppose \(0\le g\le f\), where \(g\) is bounded and supported on a set \(E\) of finite measure. If we set \(g_n(x)=\min(g(x),f_n(x))\), then \(g_n\to g\) a.e. as \(n\to \infty\). By Theorem \ref{thm:b} (Bounded convergence theorem) we have
\[
\int g_n\to \int g.
\]
Since \(g_n\le f_n\), we have \(\int f_n\le \int g_n\), so that
\[
\int g\le \liminf_{n\to\infty}\int f_n.
\]
Taking the supremum over all \(g\) yields the desired inequality.

\end{lemma}

\begin{corollary}
Suppose \(f\) is a non-negative measurable function, and \(\{f_n\}\) a sequence of non-negative measurable functions with \(f_n(x) \le f(x)\) and \(f_n(x)\to f(x)\) for a.e. \(x\). Then
\[
\lim_{n\to\infty}\int f_n=\int f.
\]
\end{corollary}

\begin{corollary}[Monotone convergence theorem]
\protect\hypertarget{cor:m}{}\label{cor:m}Suppose \(\{f_n\}\) a sequence of non-negative measurable functions with \(f_n(x) \nearrow f(x)\). Then
\[
\lim_{n\to\infty}\int f_n=\int f.
\]
\end{corollary}

\begin{corollary}
\protect\hypertarget{cor:sm}{}\label{cor:sm}Consider a series \(\sum_{k=1}^{\infty}a_k(x)\), where \(a_k(x)\ge 0\) is measurable for every \(k\ge 1\). Then
\[
\int \sum_{k=1}^{\infty}a_k(x)\ dx =\sum_{k=1}^{\infty}\int a_k(x)\ dx.
\]
\end{corollary}

\subsection{General form}\label{general-form}

In this case, we define the \textbf{Lebesgue integral} of \(f\) by
\[
\int f =\int f^+-\int f^-.
\]

\begin{proposition}
The integral of Lebesgue integrable functions is linear, additive, monotonic, and satisfies the triangle inequality.
\end{proposition}

\begin{proposition}
\protect\hypertarget{prp:uc}{}\label{prp:uc}

Suppose \(f\) is integral on \(\mathbb{R}^d\). Then for every \(\epsilon>0\):

\begin{enumerate}
\def\labelenumi{\arabic{enumi}.}
\tightlist
\item
  There exists a set of finite measure \(B\) (a ball, for example) such that
  \[
  \int_{B^c}|f|<\epsilon.
  \]
\item
  There is a \(\delta>0\) such that
  \[
  \int_E|f|<\epsilon\quad \text{ whenever }m(E)<\delta.
  \]

  \emph{Proof.}

  Assume that \(f\ge 0\): 1. \(B_N=(-N,N)\); 2. \(E_N=\{x:f(x)\le N\}\).
\end{enumerate}

\end{proposition}

\begin{theorem}[Dominated convergence theorem]
\protect\hypertarget{thm:dc}{}\label{thm:dc}\leavevmode

Suppose \(\{f_n\}\) is a sequence of measurable functions such that \(f_n\to f\) a.e. \(x\) as \(n\) tends to infinity. If \(f_n(x)\le g(x)\), where \(g\) is integrable, then
\[
\int |f_n-f|\to 0 \quad \text{ as }n\to \infty,
\]
and consequently
\[
\int f_n \to \int f.
\]

\emph{Proof.}

For each \(N\ge 0\) let \(E_N=\{x:|x|\le N, g(x)\le N\}\).
\[
\begin{aligned}
\int|f_n-f|&=\int_{E_N}|f_n-f|+\int_{E_N^c}|f_n-f|\\
&\le \int_{E_N}|f_n-f|+2\int_{E_N^c}g\\
&\le \epsilon +2\epsilon
\end{aligned}
\]
for all large \(n\). This prove the theorem.

\end{theorem}

\subsection{Complex-valued functions}\label{complex-valued-functions}

The collection of all complex-valued integrable functions on a measurable subset \(E\subset \mathbb{R}^d\) forms a vector space over \(\mathbb{C}\).

\section{\texorpdfstring{The space \(L^1\) of integrable functions}{The space L\^{}1 of integrable functions}}\label{the-space-l1-of-integrable-functions}

For any integrable function \(f\) on \(\mathbb{R}^d\) we define the \(L^1\)-norm of \(f\),
\[
\|f\|=\|f\|_{L^1}=\|f\|_{L^1(\mathbb{R}^d)}=\int_{\mathbb{R}^d}|f|.
\]

\begin{proposition}

Suppose \(f\) and \(g\) are two functions in \(L^1(\mathbb{R}^d)\).

\begin{enumerate}
\def\labelenumi{\arabic{enumi}.}
\tightlist
\item
  \(\|af\|=|a|\|f\|\) for all \(a\in\mathbb{C}\).
\item
  \(\|f+g\|\le \|f\|+\|g\|\).
\item
  \(\|f\|=0\) iff \(f=0\) a.e.
\item
  \(d(f,g)=\|f-g\|\) defines a metric on \(L^1(\mathbb{R}^d)\).
\end{enumerate}

\end{proposition}

\begin{theorem}[Riesz-Fischer]
\protect\hypertarget{thm:rf}{}\label{thm:rf}The vector space \(L^1\) is complete in its metric.
\end{theorem}

\begin{corollary}
If \(\{f_n\}_{n=1}^\infty\) converges to \(f\) in \(L^1\), the nthere exists a subsequence \(\{f_{n_k}\}_{k=1}^\infty\) such that
\[
f_{n_k}(x)\to f(x)\quad a.e. x.
\]
\end{corollary}

We say that a family \(\cal G\) of integrable functions is \textbf{dense} in \(L^1\) if for ant \(f\in L^1\) and \(\epsilon>0\), there exists \(g\in\cal G\) so that \(\|f-g\|<\epsilon\).

\begin{theorem}
\protect\hypertarget{thm:d}{}\label{thm:d}

The following families of functions are dense in \(L^1(\mathbb{R}^d)\):

\begin{enumerate}
\def\labelenumi{\arabic{enumi}.}
\tightlist
\item
  The simple functions.
\item
  The step functions.
\item
  The continuous functions of compact support.
\end{enumerate}

\end{theorem}

\subsection{Invariance properties}\label{invariance-properties}

\subsection{Translations and continuity}\label{translations-and-continuity}

\begin{proposition}
\leavevmode

Suppose \(f\in L^1(\mathbb{R}^d)\). Then
\[
\|f_h-f\|\to 0\quad \text{ as }h\to 0.
\]

\emph{Proof.}

By Theorem \ref{thm:d}, find a continuous function with compact support to approximate \(f\).

\end{proposition}

\section{Fubini's theorem}\label{fubinis-theorem}

\section{A Fourier inversion formula}\label{a-fourier-inversion-formula}

\section{Exercise}\label{exercise-1}

\section{Problem}\label{problem-1}

:::

\chapter{Differentiation and Integration}\label{ch3}

\section{Differentiation of the integral}\label{differentiation-of-the-integral}

  \bibliography{book.bib,packages.bib}

\end{document}
